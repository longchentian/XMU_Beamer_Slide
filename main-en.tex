% !TeX encoding = UTF-8
% !BIB TS-program = biber
% !TeX TS-program = xelatex
% This is file `main-en.tex'.
% Modify by Chentian Long.
% Version: 2024/07/29 v1.3e (Original Version: 2024/04/16 v1.3c by Linrong Wu) .
% 本文件为 UNIV_Beamer_Slide-demo 主文件源文件.
% !使用前请阅读用户手册.

% ================ %
%      导言区      %
% ================ %
\documentclass[hyperref, UTF8, CJK]{beamer}
%\special{dvipdfmx:config z 0}

% --------UNIV Beamer 模板宏包--------
% ----------------
\usetheme[
	CodeDisplay=minted, 
	LanguageMode=en, 
	MintedStyle=dracula,
	ColorDisplay=BSblue, 
	Background=false
]{univ}%, BIBMode=none
%\usetheme[
%LanguageMode=en, 
%ColorDisplay=BSblue, 
%Background=false
%]{univ}
%, BIBMode=none
% --------宏包调用--------
% ----------------
\usepackage{transparent}
\usepackage{array}
\usepackage{algorithm,algorithmic}
\usepackage{amsmath,amsfonts,amssymb} % math equations, symbols
\usepackage[english]{babel}
\usepackage{color}      % color content
\usepackage{url}        % hyperlinks
\usepackage{multicol,multirow}
\usepackage{ulem} % ulem: 添加线.
\usepackage{booktabs}
\usepackage{cprotect}
\usepackage{makecell}
\usepackage{listings}
\usepackage{subcaption}
\usepackage{varioref,cleveref}
%\usepackage[active,tightpage]{preview}
%\PreviewEnvironment{univcode}

% --------newcommand 区--------
% 建议在此定义常用命令.
% ----------------
\newcommand{\fverb}[1]{\texttt{#1}}

% --------封面信息输入--------
% [<in footline>], {<in title page>} 方括号内容显示在页脚, 花括号内容为全称显示在封面.
% ----------------
\title[A brief example in English for UNIV Beamer Theme]{A brief example in English}
\subtitle{For UNIV Beamer Theme} % subtitle 未设置页脚显示项, 请在 title 中设置.
\author[Chentian~Long , Linrong~Wu$^\dagger$]{Chentian~Long $^{1}$, Linrong~Wu$^{2\dagger}$}
\institute{%
	$^1$ School of Aerospace Engineering, Xiamen University, Xiamen, China\\
	$^2$ Business School, Sichuan University, Chengdu, China\\
	\textit{lr.wu.interact@outlook.com}
}
\date{\today}

% ---------------- %
%      正文区      %
% ---------------- %
\begin{document}

% --------总目录--------
% 可注释.
% ----------------
%	\begin{frame}{目录}
%		%\transfade%淡入淡出 
%		\tableofcontents % 显示目录.
%	\end{frame}

% --------节: 介绍--------
% ----------------
\section{Introduction}
\subsection{The Project}
\begin{frame}{Info.}
	\faMailForward\enspace{\color{univblue}lr.wu.interact@outlook.com}
	\faGithub\enspace{\color{univblue}\url{https://github.com/FvNCCR228/UNIV_Beamer_Slide-demo}}

	
\end{frame}

% --------节: 区块示例--------
% ----------------
\section{Blocks}
\subsection{Math Blocks}
\begin{frame}[fragile,allowframebreaks]{Math Blocks}
	\begin{univtheorem}{A Theorem}[theorem1]
		\begin{equation}
		\dfrac{1}{n} \sum_{k=1}^{n} X_k - \dfrac{1}{n} \sum_{k=1}^{n} E(X_k) \stackrel{\;P\;}{\longrightarrow} 0
		\end{equation}
	\end{univtheorem}
	\begin{univproof}{}
		A proof block.
	\end{univproof}
	\begin{univexample}{An Example}[example1]
		An example block.
	\end{univexample}
	\begin{univalgorithm}{An Algorithm}[algorithm1]
		\begin{algorithmic}[1]
			\REQUIRE \LaTeX{}
			\ENSURE Computer
			\STATE ST
			\STATE A
			\STATE TE
			\RETURN Beamer
		\end{algorithmic}
	\end{univalgorithm}
	\begin{univdefinition}{A Defintion}[defintion1]
		A definition block.
	\end{univdefinition}
	\begin{univaxiom}{An Axiom}[axiom1]
		An axiom block. Reference to~\Vref{def:defintion1}
	\end{univaxiom}
	\begin{univproperty}{A Property}[property1]
		A property block. Reference to~\Cref{axio:axiom1}
	\end{univproperty}
	\begin{univproposition}{A Proposition}[proposition1]
		A proposition block. Reference to~\vref{prope:property1}
		\begin{equation}
		\Delta x \Delta p \geq \dfrac{h}{4\pi}
		\end{equation}
		其中$h$为普朗克常数.
	\end{univproposition}
	\begin{univlemma}{A lemma}[lemma1]
		A lemma block. Reference to~\cref{propo:proposition1}
	\end{univlemma}
	\begin{univcorollary}{A Corollary}[corollary1]
		A corollary block.
	\end{univcorollary}
	\begin{univremark}{}[remark1]
		A remark block.
	\end{univremark}
	\begin{univcondition}{A Condition}[condition1]
		A condition block.
	\end{univcondition}
	\begin{univconclusion}{A Conclusion}[conclusion1]
		A conclusion block.
	\end{univconclusion}
	\begin{univassumption}{An Assumption}[assumption]
		An assumption block.
	\end{univassumption}
\end{frame}

\begin{frame}{A Stared Block}
\begin{univtheorem}*<1-4>[after title=(after title: Theorem)]{A Stared Theorem Block}[staredblock1]
	\begin{itemize}[<+->]
		\item One
		\item Two~\only<2>{Two}
		\item \alert<3>{Three}
		\item Four
	\end{itemize}
\end{univtheorem}
\begin{univtheorem}<2-5>[after title=(after title: Theorem)]{Another Stared Theorem Block}*[staredblock2]
	\begin{itemize}
		\item Five
		\item Six~\only<3>{Six}
		\item \alert<3>{Seven}
		\item Eight
	\end{itemize}
\end{univtheorem}
\end{frame}
%%重要事情说三遍 没有配置python环境的这部分不要用
%%重要事情说三遍 没有配置python环境的这部分不要用
%%重要事情说三遍 没有配置python环境的这部分不要用
%% 代码高亮部分

%\subsection{Source Code Block}
%\begin{frame}[fragile]{Source Code Block}{With frame option ``fragile''}
%	\onslide<2>
%	\begin{univcode}{A Cpp Program.}[cppcode]{c}
%		#include <iostream>
%		int main()
%		{
%			std::cout << "Hello World!" << std::endl;
%			std::cin.get();
%		}
%	\end{univcode}
%	\onslide<1>
%	\begin{univcode}{A Python Program.}[pythoncode]{python}
%		for i in range(1,5):
%			for j in range(1,5):
%				for k in range(1,5):
%					if( i != k ) and (i != j) and (j != k):
%						print (i,j,k)
%	\end{univcode}
%\end{frame}
%
%\begin{frame}[fragile]{A Stared Source Code Block}
%	\begin{univcode*}{A Stared Block.}[staredblock3]{c}
%		#include <iostream>
%		int main()
%		{
%			std::cout << "Hello World!" << std::endl;
%			std::cin.get();
%		}
%	\end{univcode*}
%	\begin{univcode}{Another Stared Theorem Block.}*[pythoncode]{python}
%		for i in range(1,5):
%			for j in range(1,5):
%				for k in range(1,5):
%					if( i != k ) and (i != j) and (j != k):
%						print (i,j,k)
%	\end{univcode}
%\end{frame}
%
%\begin{frame}[fragile]{Highlight Line}
%	\begin{univcode}{Highlight Line.}[Highlight1]{c}[highlightlines={2,5}]
%		#include <iostream>
%		int main()
%		{
%			std::cout << "Hello World!" << std::endl;
%			std::cin.get();
%		}
%	\end{univcode}
%	\begin{univcode}{Highlight Line.}[pythoncode]{python}[highlightlines={2-3,5}]
%		for i in range(1,5):
%			for j in range(1,5):
%				for k in range(1,5):
%					if( i != k ) and (i != j) and (j != k):
%						print (i,j,k)
%	\end{univcode}
%  refer \vref{code:Highlight1,code:pythoncode}
%\end{frame}
%
%\begin{frame}[fragile]{\LaTeX{} Comment}{Escapeinline}
%	If you wanna add comments to the back of the line, it is recommended to use the corresponding language comment directly.
%	\begin{univcode}{Comment.}[Comment1]{c}
%		#include <iostream>
%		int main()
%		{// $\pi$
%			std::cout << "Hello World!" << std::endl; |# \textsf{\LaTeX{} out hEllo wOrld}|
%			|\colorbox{univgreen!60}{$\sum_\pi^\phi \alpha + \Gamma$}| std::cin.get(); 
%		}
%	\end{univcode}
%	\begin{univcode}{Comment}[Comment2]{python}'@@'
%		for i in range(1,5):
%			for j in range(1,5): @$\sum_\pi^\phi \alpha + \Gamma$@
%				for k in range(1,5): # $\sum_\pi^\phi \alpha + \Gamma$
%					if( i != k ) and (i != j) and (j != k):
%						print (i,j,k)
%	\end{univcode}
%\end{frame}
%
%\begin{frame}[fragile]{Overlay \& Label}{Escapeinline}
%	\begin{univcode}{Comment.}[Escapeinline1]{c}
%		#include <iostream>
%		int main()
%		{
%			std::cout << "Hello World!" << std::endl; // \only<1>{Value 1}\only<2>{Value 2}
%			std::cin.get();
%		}
%	\end{univcode}
%	\begin{univcode}{Comment}[Escapeinline2]{python}[highlightlines={4}]'@@'
%		for i in range(1,5):
%			for j in range(1,5): 
%				for k in range(1,5):
%					if( i != k ) and (i != j) and (j != k): @\label{line:qg}@
%						print (i,j,k)
%	\end{univcode}
%  Reference to Line~\ref{line:qg}, the if statement.
%\end{frame}
%
%\begin{frame}[fragile]{Source Code From File}
%	\univcodeinput{Source Code From File}{c}{A cpp.cpp}
%\end{frame}

% --------节: 参考文献--------
% ----------------
% \section{Reference}
% \subsection{Reference}
% \begin{frame}[allowframebreaks]{Reference}
% 	\nocite{*}
% 	\printbibliography[heading=none]
% \end{frame}

% --------节: 致谢--------
% ----------------
% \section{Acknowledgement}
% \subsection{Acknowledgement}
\begin{frame}
	\centering
	\Huge Thanks!
\end{frame}

\end{document}

%% End of file `main-en.tex'.